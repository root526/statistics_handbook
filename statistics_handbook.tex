\documentclass[11pt]{ltjsarticle}
% \usepackage{luatexja}
\usepackage[top=30truemm,bottom=30truemm,left=25truemm,right=25truemm]{geometry}
\usepackage{amsmath}
\usepackage{amsthm}
  \theoremstyle{definition}
  \newtheorem{theorem}{定理}[section]
  \newtheorem{definition}[theorem]{定義}
  \newtheorem{lemma}[theorem]{補題}
  \newtheorem{corollary}[theorem]{系}
  \newtheorem{proposition}[theorem]{命題}
  \newtheorem*{remark}{注意}
  \newtheorem{example}{例}
\usepackage{amssymb}
\usepackage[stable]{footmisc}
% \usepackage[all]{xy}
% \usepackage{graphicx}
%\begin{figure}[t]
%  \begin{center}
%    \includegraphics[width=\linewidth,pagebox=cropbox,clip]{}
%    \caption{}
%    \label{}
% \end{center}
%\end{figure}
\usepackage{here}
\usepackage{enumerate}%\begin{enumerate}[括弧の種類]
\usepackage{mathrsfs}%\mathscr{}
\usepackage{braket} %\bra{}, \ket{}, \braket{ | | }
\usepackage[
    luatex,
    pdfencoding=auto,
    colorlinks=true,
    linkcolor=blue,
    setpagesize=false,
    bookmarks=true,
    bookmarksnumbered=true
]{hyperref}

%color
\usepackage{color}
\usepackage[dvipsnames]{xcolor}

%tikz
\usepackage{tikz}
\usetikzlibrary{arrows,arrows.meta,intersections, calc,positioning,decorations.pathreplacing,decorations.pathmorphing,shapes}
\usetikzlibrary{patterns}
\usetikzlibrary{decorations.markings}
\usetikzlibrary{knots}





%%%%%%%%%%%%%%%%%%%%%%% mathcal %%%%%%%%%%%%%%%%%%%%%%%%%%%
\newcommand{\CA}{\mathcal{A}}
\newcommand{\CB}{\mathcal{B}}
\newcommand{\CC}{\mathcal{C}}
\newcommand{\CD}{\mathcal{D}}
\newcommand{\CE}{\mathcal{E}}
\newcommand{\CF}{\mathcal{F}}
\newcommand{\CG}{\mathcal{G}}
\newcommand{\CH}{\mathcal{H}}
\newcommand{\CI}{\mathcal{I}}
\newcommand{\CJ}{\mathcal{J}}
\newcommand{\CK}{\mathcal{K}}
\newcommand{\CL}{\mathcal{L}}
\newcommand{\CM}{\mathcal{M}}
\newcommand{\CN}{\mathcal{N}}
\newcommand{\CO}{\mathcal{O}}
\newcommand{\CP}{\mathcal{P}}
\newcommand{\CQ}{\mathcal{Q}}
\newcommand{\CR}{\mathcal{R}}
\newcommand{\CS}{\mathcal{S}}
\newcommand{\CT}{\mathcal{T}}
\newcommand{\CU}{\mathcal{U}}
\newcommand{\CV}{\mathcal{V}}
\newcommand{\CW}{\mathcal{W}}
\newcommand{\CX}{\mathcal{X}}
\newcommand{\CY}{\mathcal{Y}}
\newcommand{\CZ}{\mathcal{Z}}


%%%%%%%%%%%%%%%%%  notations  %%%%%%%%%%%%%%%%%%

\newcommand{\inn}[2]{\langle#1,#2\rangle}%inner product
\newcommand{\norm}[1]{\|#1\|}%norm
\newcommand{\N}{\mathbb{N}}%natural number
\newcommand{\R}{\mathbb{R}}%real number
\newcommand{\C}{\mathbb{C}}%complex number
\newcommand{\Q}{\mathbb{Q}}%rational number
\newcommand{\Z}{\mathbb{Z}}%integer
\newcommand{\imineq}[2]{\vcenter{\hbox{\includegraphics[height=#2ex]{#1}}}}
\newcommand{\BS}{\mathbb{S}}
\newcommand{\BB}{\mathbb{B}}
\renewcommand{\d}{\mathrm{d}}
\renewcommand{\Set}[2]{\left\{#1\;\middle|\;#2\right\}}




\newcommand{\Smat}[2]{{\mathcal{S}_{#1}}^{#2}}
\newcommand{\Sdag}[2]{{\left(\mathcal{S}^{\dagger}\right)_{#1}}^{#2}}
\newcommand{\Tdag}[2]{{\left(\mathcal{T}^{\dagger}\right)_{#1}}^{#2}}
\newcommand{\Tmat}[2]{{\mathcal{T}_{#1}}^{#2}}
\newcommand{\fusion}[3]{{\mathcal{N}_{#1#2}}^{#3}}
\newcommand{\sph}{\mathbb{S}}
\newcommand{\torus}{\mathbb{T}}
\newcommand{\ball}{\mathbb{B}}
\newcommand{\disk}{\mathbb{D}^2}
\newcommand{\lie}[1]{\mathfrak{#1}}
\newcommand{\aff}[1]{\widehat{\mathfrak{#1}}} %affine lie algebra
\newcommand{\rep}[1]{\mathbf{#1}}
\newcommand{\nc}{G_{\text{N}}} %Newton constant
\newcommand{\PI}[1]{\mathcal{D}#1\,}
\newcommand{\tm}[2]{\tau^{#1|#2}} %transition matrix
\newcommand{\ttm}[2]{\widetilde{\tau}^{#1|#2}} %unnormalized transition matrix
\newcommand{\vir}{\mathrm{Vir}} %Virasoro algebra





%%%%%%%%%%%%%%%%%% math operators %%%%%%%%%%%%%%%%%%%%%%%%
% \newcommand{\tr}{\mathrm{tr}}%tr
% \newcommand{\Tr}{\mathrm{Tr}}%Tr
% \renewcommand{\d}{\mathrm{d}}
% \newcommand{\ad}{\mathrm{ad}}
% \newcommand{\Ad}{\mathrm{Ad}}

\DeclareMathOperator{\tr}{tr}
\DeclareMathOperator{\Tr}{Tr}
\DeclareMathOperator{\ad}{ad}
\DeclareMathOperator{\Ad}{Ad}
\DeclareMathOperator{\E}{E}
\DeclareMathOperator{\Var}{Var}

\let\Re\relax
\DeclareMathOperator{\Re}{Re}
\let\Im\relax
\DeclareMathOperator{\Im}{Im}

%%%%%%%%%%%%%%%%%%%%%% comment %%%%%%%%%%%%%%%%%%%%%%%%
\newcommand{\cmt}[1]{\textcolor{red}{\textbf{(#1)}}}





%%%%%%%%%%%%%%%%%%%%%% 数式番号 %%%%%%%%%%%%%%%%%%%

\makeatletter
\renewcommand{\theequation}{%
\thesection.\arabic{equation}}
\@addtoreset{equation}{chapter}
\makeatother




\begin{document}

\tableofcontents

\section{便利な定理まとめ}
\subsection{推定論}
\begin{theorem}\label{th:UMVU1}
    $\hat{\theta}$が不偏推定量であるとする。$\hat{\theta}$がCram\'{e}r-Raoの下限
    \begin{align}\label{CRinf}
        \Var_\theta[\hat{\theta}] = \frac{1}{nI(\theta)}
    \end{align}
    を満たすならば、$\hat{\theta}$は一様最小分散不偏(UMVU)推定量である。ここで、
    \begin{align}
        I(\theta) = \E_\theta\left[\left(\frac{\partial \log f(X|\theta)}{\partial \theta}\right)^2\right]
    \end{align}
    はFisher情報量である\footnote{Fisher情報量の定義には、$\log f(X)$のところを$n$サンプルの対数尤度関数で定義するものもあるので注意(例:竹村の$I_n(\theta)$)。}。
\end{theorem}

このCram\'{e}r-Raoの下限\eqref{CRinf}を用いたUMVUの判定方法は、あくまで十分条件を与えるのみであることに注意。一方、以下の完全十分統計量による判定方法はUMVU推定量の完全な構成法を与える。

\begin{theorem}
    $T$を完全十分統計量とする。$\hat{\theta}$が不偏推定量であるとき、
    \begin{align}
        \delta^*(T) :=\E[\hat{\theta}|T]
    \end{align}
    はUMVU推定量である。
\end{theorem}
\section{正規分布}
本節では、正規分布に関する推定量の基本的事項についてのまとめ・導出を行う。

実数値の確率変数$X$がパラメータ$\mu, \sigma$をもつ正規分布の確率密度関数
\begin{align}
    f(x|\mu, \sigma^2) = \frac{1}{\sqrt{2\pi \sigma^2}}\exp\left(-\frac{(x - \mu)^2}{2\sigma^2}\right)
\end{align}
に従うとき、$X\sim N(\mu, \sigma^2)$と表記する。

\subsection{1標本問題の点推定}
$X_1,\ldots,X_n \sim N(\mu, \sigma^2)$, i.i.d. とする。また、$\bar{X} := \frac{1}{n}\sum_{i=1}^n X_i$とする。
\begin{proposition}
    パラメータ$\mu,\sigma^2$の推定量に関して、以下が成り立つ。
    \begin{enumerate}[(1)]
        \item $\sigma^2$が既知か未知かにかかわらず
        \begin{align}
            \hat{\mu} := \bar{X} = \frac{1}{n}\sum_{i=1}^n X_i
        \end{align}
        は$\mu$のUMVU推定量かつ最尤推定量である。
        \item $\mu$が既知の場合、
        \begin{align}
            \hat{\sigma}^2 :=\frac{1}{n}\sum_{i=1}^n(X_i - \mu)^2
        \end{align}
        は$\sigma^2$のUMVU推定量かつ最尤推定量である。
        \item $\mu$が未知の場合、$\sigma^2$のUMVU推定量および最尤推定量はそれぞれ
        \begin{align}
            \hat{\sigma}_{\text{UMVU}}^2 :=\frac{1}{n-1}\sum_{i=1}^n(X_i-\bar{X})^2\, , \qquad \hat{\sigma}_{\text{MLE}}^2 :=\frac{1}{n}\sum_{i=1}^n(X_i-\bar{X})^2
        \end{align}
        で与えられる。
    \end{enumerate}
\end{proposition}
\begin{proof}
    \begin{enumerate}[(1)]
        \item $\E[\hat{\mu}]=\mu$が成り立つから、$\hat{\mu}$は$\mu$の不偏推定量。
        \begin{align}
            \log f(X|\mu,\sigma^2) = -\frac{1}{2}\log(2\pi\sigma^2) - \frac{(X-\mu)^2}{2\sigma^2}
        \end{align}
        より、$\mu$に関するFisher情報量は
        \begin{align}
            I(\mu) = \E\left[\frac{(X-\mu)^2}{\sigma^4}\right] = \frac{1}{\sigma^2}\, .
        \end{align}
        $X_1,\ldots,X_n$は互いに独立だから、
        \begin{align}
            \Var[\hat{\mu}] = \frac{1}{n^2}\sum_{i=1}^n\Var[X_i] = \sigma^2/n\, .
        \end{align}
        よって定理\ref{th:UMVU1}より$\hat{\mu}$はUMVU推定量である。

        次に最尤推定量が$\hat{\mu}$になることを示す。対数尤度は、
        \begin{align}
            \ell(\mu, \sigma | X^n) = -\frac{n}{2}\log(2\pi\sigma^2) - \sum_{i=1}^n\frac{(X_i-\mu)^2}{2\sigma^2}
        \end{align}
        これを$\mu$について最大化すると、
        \begin{align}
            \hat{\mu}_{\text{MLE}} = \frac{1}{n}\sum_{i=1}^n X_i
        \end{align}
        となり、これは$\hat{\mu}$に一致。

        \item $X_1,\ldots,X_n$は互いに独立であることから$\E[\hat{\sigma}^2]=\sigma^2$が成り立つため、$\hat{\sigma}^2$は$\sigma^2$の不偏推定量。$\sigma$に関するFisher情報量は
        \begin{align}
            I(\sigma) = 
        \end{align}
    \end{enumerate}
\end{proof}


\end{document}


